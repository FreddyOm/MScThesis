\chapter{Related Work} \label{cpt-related-work}

The proposed approach is an aggregation of various ideas and implementations that have been 
around for quite some time. It is of great interest to explore related work, which can provide 
context for use cases and highlight additional techniques that can be adopted additionally. 
This chapter focuses on related work addressing \emph{voxel rendering}, \emph{occlusion culling}, 
and \emph{mesh shading}, on which the proposed implementation is based. The most relevant concepts 
are revisited and discussed in more detail in Chapter \ref{cpt-technical-background}. 


\section{Voxel Representation} \label{sec-voxel-representation}

There are several ways to represent volumetric data. The simplest version includes a regular 
grid with grid indices representing unique positions within the grid. The presence of a voxel 
(\emph{voxel} = \emph{volumetric pixel}) can be encoded as a binary value, which results in 
an efficient way of storing volumetric occupation. However, empty spaces are stored in the 
same resolution as crowded spaces, which led to the introduction of octrees for volumetric 
voxel representations. The term \ac{SVO} relates to an octree that does not store any child 
data for empty sub-branches of the hierarchy. Most modern implementations rely on some of 
these basic encodings of voxel data. 


\subsection*{Efficient Sparse Voxel Octrees}

Laine and Karras \cite{Laine2010} analyze and implement a variation of the base sparse voxel Octree approach for 
efficiently raytracing voxel models. In their work, contours are used to approximate octree nodes within an \ac{SVO}. 
If the octree node is found to be approximated good enough, the subdivision to deeper levels in the octree node is 
terminated \cite{Kampe2013,Laine2010}. This way, \ac{SVO}s can be significantly reduced in size, which benefits 
the memory footprint of the application. \\

\noindent
The efficient \ac{SVO}s feature a similar approach to this work in that octree nodes are approximated for faster 
processing times. While Laine et al. use contours for their approximations, this work uses whole octree nodes 
to approximate volumes. \\

\noindent
However, Laine et al. use their implementation for raytraced voxel rendering, and the approximations are only 
targeting model boundaries instead of volumes. Therefore, the approximation using contours is not aiming for the 
same optimization that the approximation in this work does. While the first one refers to optimizing ray casting, 
the latter refers to optimizing a depth prepass computation. Nevertheless, both approaches share the aspect of 
reducing octree complexity by approximating parts of the tree hierarchy.


\subsection*{Sparse Voxel Directed Acyclic Graphs}

Kämpe et al. \cite{Kampe2013} propose the use of their \emph{high-resolution sparse voxel \ac{DAG}}. This approach 
is tailored to high voxel resolutions and data that might not fit into memory all at once. Their data structure can 
be easily split up into several subtrees, which in turn can be loaded individually. This approach seems promising 
for large scenes and use cases, which require the streaming of data. They propose to structure the data in a way that 
enables equal nodes of an octree hierarchy to be reduced to only one instance, eliminating redundant parts of the 
hierarchy and reducing the amount of used memory immensely. \\

\noindent
This approach relates to this implementation in how large voxelized scenes can be efficiently loaded and stored in 
memory. Although this work doesn't explicitly rely on sparse voxel \ac{DAG}s, sparse voxel octrees are used in the 
provided implementation. Additionally, the proposed implementation can be optimized in the future by implementing 
sparse voxel \ac{DAG}s, enabling the handling of meshes with even higher resolution. \\

\noindent
Sparse voxel \ac{DAG}s are not commonly used in rasterized rendering pipelines and store volumetric data in a binary 
format instead of storing positional data. This introduces additional complexity to the pipeline, which might result in 
better runtime performance but is also less intuitive when dealing with nodes and individual voxels. Sparse voxel 
\ac{DAG}s mainly aim to provide a smaller memory footprint for raytraced applications while maintaining high-resolution 
voxel meshes. \\

\noindent
Compared to this work's approach, sparse voxel \ac{DAG}s are able to resolve voxel data more efficiently and are 
better suited for higher resolution meshes that might exceed memory limits in a raw data format. However, their 
primary usage is ray-traced rendering, which is not what this work is concerned with. While sparse voxel \ac{DAG}s 
could generally be applied to rasterized rendering, this work uses a custom sparse voxel octree optimized for 
traversal and easy coordinate extraction. 


\section{Occlusion Culling}

The goal in occlusion culling is to remove objects from the rendering pipeline that are invisible due to 
occlusion. Traditionally, a lot of occlusion culling algorithms rely on \ac{CPU} computation of visibility, 
although recent approaches more often rely on \ac{GPU}-based algorithms for reasons of performance. 


\subsection*{Hierarchical Z-Buffering}

Greene et al. \cite{Greene93,Greene95} proposed the first \ac{HZB} algorithm with the capability to use screen space 
data to cull meshes in large and densely populated scenes. The idea originated in the context of a forward-rendered 
pipeline, making use of the \ac{CPU} for culling instances. They propose to use a spatial container like an octree to 
enable drawing the scene in a rough front-to-back order. Each mesh is individually drawn to a depth buffer resource, 
starting with an empty one in the beginning and populating it as more meshes are drawn. The buffer resource also 
maintains a mip chain that is updated when the full-resolution depth buffer is updated. Each time a mesh is rendered 
to the depth buffer, it is checked against the depth hierarchy, starting with the coarsest mip level. This way, near 
and big objects are likely to occlude small, far objects, resulting in an efficient way to reject invisible meshes.\\

\noindent
Although this technique was intended for use on the \ac{CPU}, it can be easily adapted to work on \ac{GPU} resources 
without the need of a write-back to expose the data to \ac{CPU} memory. The approach by Greene et al. is the basis for 
modern \ac{GPU}-based algorithms and a major influence for the work at hand. It uses the same principle of hierarchical 
visibility testing and maintaining a depth buffer. \\

\noindent
However, more modern approaches are used to adapt this algorithm for \ac{GPU}-friendly design, like computing the depth 
in a pre-pass, using only the best occluders instead of all possible meshes, and disabling color writes completely. This 
also limits the amount of recomputations of the mip-chain since all occluders are only drawn to the depth buffer once per 
frame instead of scaling with the number of visible meshes.


\subsection*{Portal- and Cell-Based Occlusion Culling}

Luebke et al. \cite{Luebke1995} present a less exact but much simpler approach to the determination
of a \ac{PVS}. Their approach relies on subdividing space into cells, which are interconnected by 
portals. The portals are represented by boundary boxes, which are projected into screen space to 
create a distinct testing boundary for objects in the adjacent cell. If the object's boundaries 
intersect with the portal's boundaries in screen space, they are potentially visible. If they do 
not intersect whatsoever, the object can be safely culled. This way, potentially visible meshes can 
be determined dynamically and at runtime, enabling the culling of large parts of the scene. \\

\noindent
Like similar approaches before, this cell-based technique can be used to coarsely evaluate visibility 
for large and complex scenes without complex overhead. The underlying concept was also used by 
\emph{Umbra} \cite{Umbra2024}, an occlusion culling tool used by a lot of games, such as 
\emph{Call of Duty: Ghosts} (2013), \emph{Killzone: Shadow Fall} (2013), \emph{Alan Wake} (2012), 
and many more \cite{UmbraWiki,CallOfDutyGhostsCredits,KillzoneUmbra,AlanWakeUmbra}. \\

\noindent
While \emph{Umbra} additionally makes use of a simplified depth buffer, similar to this work, the approach 
presented by Luebke et al. does not use depth information for visibility testing. However, projecting a 
bounding box to screen space has similarities to this work. A cell and portal occlusion culling algorithm 
can potentially be used in both voxel scenes and traditional non-voxel scenes but requires a 
predetermination of portals, which connect different areas of the scene. This means that the approach would 
not satisfy the requirement of dynamically adapting to a changing environment at run time. \\


\subsection*{Two-Pass Occlusion Culling}

Aaltonen et al. \cite{Aaltonen2015} present a \ac{GPU}-based two-pass occlusion culling technique based on 
the \ac{HZB} approach proposed by Greene et al. \cite{Greene93,Greene95}. This technique additionally features 
depth reprojection, which makes for a good occlusion culling technique in games with large occluders or interiors. 
First, artist-picked \emph{best occluders} are drawn to the depth buffer in a depth pre-pass. Then, projected 
bounding boxes can be used to test the visibility of smaller objects in the scene. A hierarchical z-buffer is 
used to accelerate this testing, as proposed by Greene et al. Finally, the depth buffer of the last frame can 
be reprojected to further minimize computations. This technique allows developers to push for huge draw distances 
and densely crowded environments. \\

\noindent
Similar to this work, the game uses advanced technology in its implementation, clustering geometry, and allowing 
for a \ac{GPU}-centric approach to occlusion culling. However, the game doesn't use a volumetric representation 
but the standard boundary representation for its world. Furthermore, the best occluders are provided in different 
ways. Aaltonen et al. \cite{Aaltonen2015} propose the use of a hand-picked set of large, static meshes, while this 
work relies on a simple approximation of the voxel layout. This way, the latter implementation can dynamically 
provide the best occluders during runtime, adapting the occlusion capabilities with the addition or removal of 
voxels in the scene.


\subsection*{Masked Software Occlusion Culling}

Hasselgren et al. \cite{Hasselgren2016} present a novel way of implementing occlusion culling relying on hardware 
occlusion queries while optimizing the queries using \ac{CPU} \ac{SIMD} acceleration. The basic algorithm is very 
similar to common \ac{CPU}- or \ac{GPU}-computed \ac{HZB}, with two major alterations to the implementation. First, 
they make use of Intel's \ac{AVX} to compute depth values for tiles of 256 texels in parallel. The second difference 
is the way they store depth and coverage data. In contrast to a "normal" \ac{HiZ} mip-chain, Hasselgren et al. 
decouple depth and coverage data by storing a coverage mask for \begin{math}32 \times 8\end{math} pixel tiles, 
which can be efficiently computed by using the \ac{AVX} of Intel \ac{CPU}s. In their results, they show that their 
implementation is up to 3 times faster than previous occlusion culling algorithms while also providing a lower 
memory footprint. \\

\noindent
The work of Hasselgren et al. relates to this work's implementation in the general concept of occlusion culling, making 
use of projected bounding boxes, and a comparison of minimal z values. Their work shows that such an algorithm can be 
optimized for \ac{CPU} execution using \ac{SIMD} and effective data structures. \\

\noindent 
In contrast to this work, Hasselgren et al. use their algorithm for non-volumetric triangle meshes. This means that 
they make heavy use of raster occlusion queries, making rendering CPU-bound. While still efficient and possibly faster 
than \ac{HZB}, masked software occlusion culling lacks the \ac{GPU}-driven basis for this work's implementation. This 
work's approach does not work with geometry on the \ac{CPU} side at all but leaves all of the triangle computations up 
to the last possible stage of the pipeline, making the application of \ac{CPU}-accelerated occlusion queries impractical.


\subsection*{Per-Meshlet Two-Pass Occlusion Culling}

Karis et al. \cite{Karis2021} and Remedy Entertainment \cite{Remedy2023} find that using \ac{HZB} can provide great 
runtime performance for open and highly dense scenes. Both combine the \ac{HZB} with depth reprojection to further optimize 
the computation of the depth buffer. Using the mesh shading pipeline, this concept can be applied to meshlets, enabling 
per-meshlet culling. That means that small parts of the geometry can be culled, even if other parts of the same mesh are 
still visible. This optimizes the culling further so scenes can be more densely populated with meshes. \\

\noindent
The findings of Karis et al. and Remedy Entertainment are adapted for this work's implementation, which also relies on 
the use of the mesh shading pipeline. While the implementations of Karis et al. and Remedy Entertainment are based on 
a boundary representation and do not require the use of a spatial container for the culling algorithm, this 
implementation aims for a different use case. Contrary to boundary representations, the spatial properties of voxels 
can be used to approximate volumetric coverage by less complex geometry, making it more efficient to select and draw the 
best occluders. Also, this work uses the best occluders not only to occlude meshlets, which are behind the best occluders, 
but it can additionally be used to occlude parts of the best occluders themselves. This poses a new way of applying occluders 
in the context of rendering volumetric scenes.
