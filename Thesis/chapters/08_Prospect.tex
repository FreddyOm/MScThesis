\chapter{Future work} \label{cpt-future-work}

Considering the results and limitations presented, there are a few aspects that benefit from further 
research and experimentation. Some of them couldn't be consider in this work due to limitations in 
scope, others are direct consequences from the results in this work. The aspects considered for 
future work are discussed below. \\

\noindent
It might be beneficial to consider some optimizations for the implementation of the whole pipeline that 
optimize the stages even further. First of all, the voxel data layout can be optimized by using a different
approach to storing the voxel data. Instead of a position buffer and a regular grid, which were used due to 
their simplicity, Sparse Voxel \ac{DAG}s or similar scene representations could be used for optimizing 
memory usage. The efficient way of storing data, and how they implicitly manage voxel data without the 
need for a seperate position buffer, can possibly increase the performance, especially when using high 
resolution voxel models. Another optimization is the use of indirect draw calls and command lists for 
more efficient playback of \ac{GPU} commands and the possible optimization towards a completely 
\ac{GPU}-driven pipeline. In this work's implementation, there is a small number of \ac{CPU} operations 
between the depth prepass and the actual draw call, which can potentially be eliminated by using indirect 
draw calls and command lists. Finally, depth reprojection could be applied to this implementation as an 
additional optimization. Comparing the results of the adapted \ac{TPOC} with and without depth reprojection 
might provide valuable insights into the optimization potential of this approach. \\

\noindent
Another aspect to consider is a further inspection of configuration values for this work's implementation. 
For example, the threadgroup size in combination with the scene resolution and minimum octree node size 
can provide different results. A higher threadgroup size allows more voxels to be present within one octree 
node and subsequently within the whole scene. This might also change the way, how efficiently \ac{GPU} 
threads are used and can ultimately impact the overall performance in either direction. Further measurements 
can provide insights into optimal relations between the amount of \ac{GPU} threads, octree nodes and node 
sizes. \\

\noindent
The \ac{HiZ} pyramid is created sequentially in the given implementation, which means that one mip level is 
created by the compute shader before the next higher mip level is created. The \ac{HiZ} creation can therefore 
be optimized by creating all mip levels in parallel or in a small number of batches, without the need to wait 
for one compute dispatch to finish. Given this adaption of the \ac{HiZ} creation, the depth prepass might be 
even faster and less dependent on the screen resolution. \\

\noindent
The provided implementation features some visual glitches during occlusion culling as long as higher mip levels 
are used for the deoth prepass. This issue led to the assumption that the hierarchy isn't actually obligatory 
for per-meshlet culling, since no sub-voxel detail is present anyway. This assumption needs to be tested in 
order to be finally validated. If the assumption holds true, the \ac{HZB} could be made completely redundant and 
the occlusion culling algorithm can be simmplified further. \\

\noindent
Finally, the most important experiment to be done in the future is dynamically updating the voxel data and therefore 
altering the best occluders at runtime. As the experiment showed, the occlusion culling algorithm was not only 
capable of reducing the amount of meshes computed by the rendering pipeline, it was almost completely executed 
on the \ac{GPU}. This leaves some more \ac{CPU} resources for updating voxel data and \ac{GPU} buffers. In order 
to update the best occluders efficiently, the implementation would benefit from adding chunks to the scene structure. 
A future experiment could test the performance of this approach within a chunked scene, implementing physics operations 
that manipulate voxel data in realtime. 

