\chapter{Experiment} \label{cpt-experiment}

To be able to evaluate the strengths and weaknesses of the given implementation, an in-depth 
experiment is conducted and evaluated. When considering the implementation, 


\section{Experimental Evaluation} \label{sec-experimental-evaluation}

To evaluate the key differences and the runtime performance in general, several aspects have to be considered.
Unfortunatly, comparing the occlusion culling implementation to an alternative implementation is not within the 
scope of this work, so efforts need to be made in order to evaluate the pipeline and its performance from a 
conceptual point of view and compared to the plain pipeline, without any occlusion culling at all. The next  
sections highlight the major aspects which are considered for the experiment and are evaluated during the 
test runs. All aspects aim to demonstrate how the occlusion culling affects the given test scene and how the 
computations differ compared to the plain pipeline. Later, use cases are considered which can be compared to 
alternative algorithms on a high level. However, keep in mind that it is up to future measurements to quantitavely
compare the pipeline to alternative approaches. 


\subsection*{Scene Layout}

First of, the scene layout and scene description need to be considered. Any occlusion culling algorithm somehow 
relies on spatial information, be it in world space, sceen space or by using a spatial container. Consequently, 
the scene layout and how the scene changes relative to the camera are important aspects to consider. For this 
experimental setup, a moving camera is used in order to test the ability of the occlusion culling algorithm to 
adapt to different angles of the scene. 

Another aspect to consider is the model data structure. The voxel models are volumetric representations and thus, 
not hollow. This leaves a large portion of computation to be irrelevant and allows for the occlusion culling to 
optimize the data flow. Different types of models are used to challenge the culling algorithm in different ways. 
In concept, the culling should be optimal on large, dense models with even faces. Consequently, models with 
slopes and holes are tested, to see how well best occluders can be computed. Smaller models might also pose a 
challenge for the optimization, especially when the octree nodes cannot be filled completely. Also, different 
voxle resolutions are used to achieve different amounts of octree nodes in the scene. All these different 
aspects should show the strengths and weaknesses of the algorithm and how it can be optimally used in practice.


\subsection*{Timings}

Ultimately, the time spent for various computations is relevant for the selection of the best algorithm. This is 
a central tool to see how the implementation performs overall in the context of the framework used. The frame 
time is the first measurment to be considered, but not necessarily the most meaningful in a complex pipeline. 
Consequently, the timings for all major pipeline steps are considered, as well as the overall time spend on \ac{CPU} 
and \ac{GPU} computations. 



Frame time, 
time spent in different pipeline steps,
CPU time,
GPU time

When possible, compare timings to findings by \cite{Aaltonen2015} or \cite{Remedy2023}




\subsection*{Visibility and Culling}

In order to evaluate the general functionality and effectiveness of the pipeline, the voxel model is analysed 
with and without the occlusion culling applied. The amount of culled voxels is compared to the total amount of 
voxels resulting in an overview of how much geometry can be culled for different scene setups. The same is 
valid for the amount of octree nodes. They can also be compared to the amount of best occluders. The more best 
occluders present in a scene, the higher the probability to occlude other voxels or octree nodes. To evaluate 
how much can be optimized during the depth pre pass, the amount of best occluders can be compared to the amount 
of voxels representing the best occluders. Since the voxels are aggregated to larger cubes, drawing the inner 
volume of the voxel model is more efficient than drawing all the voxels individually. [@TODO: Make this sentance 
understandable]  
Finally, the actual amount of processed triangles shows the geometry processed by 
the \ac{GPU}. 

Visible voxels, culled voxels,
Overdraw, Visible Octree Nodes,
Visible Triangles, Computed Triangles,
Amount of Best Occluders,
Amount Best Occluders vs. aggregated octree nodes (how many "voxels" do I draw for best occluders instead of meshlets)


\subsection*{Additional Overhead}

Ultimately, the additional overhead needs to be evaluated compared to the 

What computations are done only for the OC algo?
Rough timings / percentage of complete pipeline times



Wieso das Experiment,
Wie gehe ich vor, warum?
Welche Schwächen haben andere OC algos, und wo kann unserer im Vergleich glänzen



\section{Measurements}

- NVIDIA NSight
- RenderDoc
- ...

\section{Experimental Environment} \label{sec-experimental-environment}

- Compare:
    - Model turntable rendered without OC
    - Model turntable rendered with OC
(   - Model turntable rendered with different OC implementation (? -> hard))


- Criteria: 
    - Frame time
    - Dispatch numbers (?)
    - Duration of depth pre-pass
        - Draw best occluders to depth buffer
        - Duration of depth hierarchy creation
    - Visible cubes
    - Visible octree nodes
    - Triangle count
    - Amount of best occluders
    - CPU time
    - GPU time
    - Amount of overdraw (heatmap if possible)


- Model variations:
    - Lucy
    - Stanford Bunny
    - Torus (for when objects have "holes")
    - Some large, wide scenery
    - Sponza        -> Ehrlich sein und T
    - ...


- Machines:
    - Laptop
    - RTX 2080 Ti
    - RTX 4090
    - Multiple devices


\section{Tests}




\section{Results}

- Results of the case study


