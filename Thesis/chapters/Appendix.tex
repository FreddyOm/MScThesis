\chapter{Appendix} \label{cpt-appendix}

- Code

\section*{CPU Frame Times}


\begin{figure}[!htbp]              % Lucy CPU times
    \begin{center}
      \begin{tikzpicture}
        \begin{axis}[
            width=\linewidth,
            height=100px,
            xlabel={Frames},
            ylabel={Frame Time (s)},
            grid,
            xmin=0,
            xmax=1868,
            ymin=0,
            ymax=0.04,
            legend style={at={(0.5,1.8)}, anchor=north, legend columns=2},
          ]
          \addplot[blue, no marks, solid] table[col sep=comma, x=frame, x expr=\thisrow{frame} * 1868 / 1864, y=time]{./plotdata/cpu/lucy_256_frameTime_pooc.csv};
          \addplot[red, no marks, solid] table[col sep=comma, x=frame, x expr=\thisrow{frame} * 1868 / 1868, y=time]{./plotdata/cpu/lucy_256_frameTime.csv};
          \legend{Per-Octree Occlsion Culling, Per-Meshlet Occlusion Culling}
        \end{axis}
      \end{tikzpicture}
    \end{center}
    \caption{The frame time measured over the course of the test animation on the \ac{CPU} side for the \emph{Lucy} scene.
    The frame times were artificially capped to 30, 60 or 120 Hz by \emph{Diligent Engine}.}
    \label{plt:lucy-256-culling-cpu-frame-time}
  \end{figure}

\begin{figure}[!htbp]              % Bunny CPU times
    \begin{center}
      \begin{tikzpicture}
        \begin{axis}[
            width=\linewidth,
            height=100px,
            xlabel={Frames},
            ylabel={Frame Time (s)},
            grid,
            xmin=0,
            xmax=1701,
            ymin=0,
            ymax=0.04,
            legend style={at={(0.5,1.8)}, anchor=north, legend columns=2},
          ]
          \addplot[blue, no marks, solid] table[col sep=comma, x=frame, x expr=\thisrow{frame} * 1701 / 1548, y=time]{./plotdata/cpu/bunny_256_frameTime_pooc.csv};
          \addplot[red, no marks, solid] table[col sep=comma, x=frame, x expr=\thisrow{frame} * 1701 / 1701, y=time]{./plotdata/cpu/bunny_256_frameTime.csv};
          \legend{Per-Octree Occlsion Culling, Per-Meshlet Occlusion Culling}
        \end{axis}
      \end{tikzpicture}
    \end{center}
    \caption{The frame time measured over the course of the test animation on the \ac{CPU} side for the \emph{Bunny} scene.
    The frame times were artificially capped to 30, 60 or 120 Hz by \emph{Diligent Engine}.}
    \label{plt:bunny-256-culling-cpu-frame-time}
  \end{figure}

\begin{figure}[!htbp]              % Torus CPU times
    \begin{center}
      \begin{tikzpicture}
        \begin{axis}[
            width=\linewidth,
            height=100px,
            xlabel={Frames},
            ylabel={Frame Time (s)},
            grid,
            xmin=0,
            xmax=1786,
            ymin=0,
            ymax=0.04,
            legend style={at={(0.5,1.8)}, anchor=north, legend columns=2},
          ]
          \addplot[blue, no marks, solid] table[col sep=comma, x=frame, x expr=\thisrow{frame} * 1786 / 1555, y=time]{./plotdata/cpu/torus_256_frameTime_pooc.csv};
          \addplot[red, no marks, solid] table[col sep=comma, x=frame, x expr=\thisrow{frame} * 1786 / 1786, y=time]{./plotdata/cpu/torus_256_frameTime.csv};
          \legend{Per-Octree Occlsion Culling, Per-Meshlet Occlusion Culling}
        \end{axis}
      \end{tikzpicture}
    \end{center}
    \caption{The frame time measured over the course of the test animation on the \ac{CPU} side for the \emph{Torus} scene.
    The frame times were artificially capped to 30, 60 or 120 Hz by \emph{Diligent Engine}.}
    \label{plt:torus-256-culling-cpu-frame-time}
  \end{figure}

\begin{figure}[!htbp]              % Terrain CPU times
    \begin{center}
      \begin{tikzpicture}
        \begin{axis}[
            width=\linewidth,
            height=100px,
            xlabel={Frames},
            ylabel={Frame Time (s)},
            grid,
            xmin=0,
            xmax=1814,
            ymin=0,
            ymax=0.04,
            legend style={at={(0.5,1.8)}, anchor=north, legend columns=2},
          ]
          \addplot[blue, no marks, solid] table[col sep=comma, x=frame, x expr=\thisrow{frame} * 1814 / 1572, y=time]{./plotdata/cpu/terrain_256_frameTime_pooc.csv};
          \addplot[red, no marks, solid] table[col sep=comma, x=frame, x expr=\thisrow{frame} * 1814 / 1814, y=time]{./plotdata/cpu/terrain_256_frameTime.csv};
          \legend{Per-Octree Occlsion Culling, Per-Meshlet Occlusion Culling}
        \end{axis}
      \end{tikzpicture}
    \end{center}
    \caption{The frame time measured over the course of the test animation on the \ac{CPU} side for the \emph{Terrain} scene.
    The frame times were artificially capped to 30, 60 or 120 Hz by \emph{Diligent Engine}.}
    \label{plt:terrain-256-culling-cpu-frame-time}
  \end{figure}

\begin{figure}[!htbp]              % Hairball CPU times
    \begin{center}
      \begin{tikzpicture}
        \begin{axis}[
            width=\linewidth,
            height=100px,
            xlabel={Frames},
            ylabel={Frame Time (s)},
            grid,
            xmin=0,
            xmax=1446,
            ymin=0,
            ymax=0.025,
            legend style={at={(0.5,1.8)}, anchor=north, legend columns=2},
          ]
          \addplot[blue, no marks, solid] table[col sep=comma, x=frame, x expr=\thisrow{frame} * 1446 / 1422, y=time]{./plotdata/cpu/hairball_256_frameTime_pooc.csv};
          \addplot[red, no marks, solid] table[col sep=comma, x=frame, x expr=\thisrow{frame} * 1446 / 1446, y=time]{./plotdata/cpu/hairball_256_frameTime.csv};
          \legend{Per-Octree Occlsion Culling, Per-Meshlet Occlusion Culling}
        \end{axis}
      \end{tikzpicture}
    \end{center}
    \caption{The frame time measured over the course of the test animation on the \ac{CPU} side for the \emph{Hairball} scene.
    The frame times were artificially capped to 30, 60 or 120 Hz by \emph{Diligent Engine}.}
    \label{plt:hairball-256-culling-cpu-frame-time}
  \end{figure}
  

\section*{Project Code}

