\chapter{Conclusions} \label{cpt-conclusion}

% Was wurde gemacht und warum ?

This work has presented an implementation for a \ac{GPU}-driven implementation of 
\ac{HZB} for voxel-based volumetric rendering. A spatial data structure is 
used to approximate the scene's inner volume and replace vast amounts of voxels 
with a small amount of cubical geometry. This way, a set of best occluders can 
be used for computing a depth prepass, which in turn is used for efficient 
occlusion culling on the \ac{GPU}. An exemplary voxel scene with over 3 million 
voxels has been approximated by only 8,256 larger voxels.  \\

% Quantitative Ergebnisse

\noindent
The measurements have shown the capability of the modern mesh shading rendering 
pipeline to increase culling efficiency by culling individual meshlets, which in 
turn leads to a lower computational load for the rasterizer and pixel shader 
stages. The experiment has provided a precise measurement that indicates a \ac{GPU} 
performance increase of about $29 \%$ when using the \ac{PMOC} configuration 
of the algorithm as compared to the more general \ac{PONOC} configuration. 
Compared to the base pipeline without occlusion culling applied to the scene, the 
performance increase ranged from $61.3 \%$ for the highly complex and irregular 
\emph{Hairball} scene up to $90.7 \%$ for the \emph{Bunny} scene. Altogether, the 
measurements indicate a better overall performance when using the modern mesh shading 
pipeline.\\

\noindent
Furthermore, the pixel shader overdraw was shown to be considerably reduced by using 
the \ac{PMOC} as compared to no occlusion culling at all or even the \ac{PONOC} 
configuration. Overdraw is mostly present on the surface of the scene, while being 
greatly reduced for otherwise inefficient camera angles. \\

\noindent
The measurements for querying the best occluders have also shown that the approach can 
generally be used to update the voxel data dynamically in real time. This is a key 
requirement for the use of the algorithm in real-time voxel-rendered games since 
manipulating the voxel data arbitrarily is a common feature in such games. The 
measurements have shown that the \ac{CPU} computations required for querying the 
best occluders are within reasonable time budgets. Further optimizations, such as 
multithreading and chunked voxel data, are expected to enhance performance. \\

\clearpage

\noindent
The approach is particularly suited for dense and dynamic scenes, such as voxel-rendered 
games or animated high-resolution models. It can significantly reduce computational load 
on the \ac{GPU} and \ac{CPU}, as well as pixel shader overdraw, especially for dense scenes with 
favorable occluder geometry. The measurements also indicate real-time capability, as 
\ac{CPU} computations for occluder updates are within acceptable limits. Further 
optimizations, such as multithreading and chunked voxel data, are expected to enhance 
performance. \\

% Einschränkungen und Herausforderungen

\noindent
The measurements have provided an overview of culling efficiency depending on several 
aspects of the algorithm's configuration. The model's characteristics have a significant 
effect on the culling efficiency. Models with high volumes and flat surfaces benefit most, 
while thin, detailed geometries or low voxel resolutions perform less efficiently due 
to fewer full octree nodes. The approach is therefore suggested for scenes that are dense 
and tight and can be expected to provide at least a reasonable amount of full octree nodes. 
The main factors to influence the performance are the number of voxels, the number of 
best occluders, and the screen resolution.\\

\noindent
Ultimately, this work has shown that voxel rendering can benefit from the latest advances 
in real-time rendering and that the use of the proposed occlusion culling algorithm 
using modern mesh shading can optimize the runtime performance as compared to a similar 
\ac{PONOC} approach or compared to not using any occlusion culling at all. \\

\noindent
Finally, this work highlights the potential of modern rendering pipelines for voxel-based 
graphics. The proposed algorithm could be used in games like \emph{Minecraft} (Mojang 
\cite{Mojang2024}, 2011) for optimizing the rendering using a \ac{GPU}-driven approach. It 
could also be extended for use on highly dynamic scenes and even voxel-based animations, 
which could be explored in more detail in the future.
