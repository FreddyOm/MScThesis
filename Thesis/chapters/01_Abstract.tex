\cleardoublepage

% Start with German abstracrt
\begin{otherlanguage}{ngerman}
\chapter*{Kurzfassung}
\addcontentsline{toc}{chapter}{Kurzfassung}

Der Hierarchical Z-Buffering Algorithmus hat innerhalb der letzten Jahre an Popularität gewonnen, 
da er verdeckte Geometrie mit Hilfe eines GPU-basierten Ansatzes effizient verwerfen kann. Dieser 
Algorithmus war für voxelbasiertes volumetrisches Rendering in seiner ursprünglichen Form nicht 
verfügbar. In dieser Arbeit wird ein GPU-basierter Ansatz für Hierarchical Z-Buffering vorgestellt, 
der in voxelbasierten volumetrischen Szenen verwendet werden kann, indem das Volumen der Szene durch 
größere Geometrie approximiert wird. Der Ansatz verwendet einen Octree, um Voxeldaten in Bezug auf 
ihre räumlichen Eigenschaften effizient zu verarbeiten, und ersetzt volle Octree-Nodes durch größere 
Geometrie für eine schnelle Berechnung des Hierarchical Z-Buffering. Dies ermöglicht nicht nur die 
Anwendung von Hierarchical Z-Buffering in dynamischen Voxelszenen, sondern führt auch zu einer 
Leistungssteigerung im Vergleich zu einer Rendering-Pipeline ohne Occlusion Culling. Diese Arbeit 
zeigt, dass die Verwendung der modernen Mesh-Shading-Pipeline die Berechnungen verbessert und zu 
einer noch besseren Culling-Effizienz und einer gesteigerten Gesamtleistung führt.

\vfill
\noindent\textbf{Stichwörter:} Mesh Shading, Real-Time Rendering, Voxel Graphics, Occlusion Culling
\vfill
\end{otherlanguage}
% Then continue with the english one.
\begin{otherlanguage}{english}
\chapter*{Abstract}
\addcontentsline{toc}{chapter}{Abstract}

The hierarchical z-buffering algorithm has recently gained popularity due to its ability to 
efficiently cull occluded geometry using a GPU-driven approach. This algorithm has not been 
available in its original form for voxel-based volumetric rendering. Here, a GPU-driven 
approach to hierarchical z-buffering is presented, for use in voxel-based volumetric scenes 
by approximating the scene's volume with larger geometry. The approach uses an octree to 
efficiently schedule voxel data with respect to their spatial characteristics and replaces 
full octree nodes with larger geometry for fast computation of the hierarchical z-buffer. 
This not only allows the application of hierarchical z-buffering to a dynamic voxel scene, 
but also results in a performance increase compared to a rendering pipeline without occlusion 
culling. This work shows that the use of the modern mesh shading pipeline improves computations 
and leads to even better culling efficiency and overall performance.

\vfill
\noindent\textbf{Keywords:} mesh shading, real-time rendering, voxel graphics, occlusion culling
\vfill
\end{otherlanguage}
