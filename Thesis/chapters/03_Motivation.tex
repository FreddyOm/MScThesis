\chapter{Motivation} \label{cpt-motivation}


Ever since the early days of computer graphics, both hard- and software have rapidly evolved
alongside the creative and challenging use cases provided by developers, scientists and others.
Many technical increments relied on innovation, especially the advent of new capable hardware.
One example is the \ac{GPU} itself, which is now considered the heart of modern graphics 
processing and is widely used in fields like science, artificial intelligence, games, and pretty 
much all graphics related applications. Before 1995, there already had been a lot of iterations 
on specialized graphics hardware, often focused on video formatting or color output operations
\cite{Singer2023}. [@TODO: Last sentence is redundant]\\

\noindent [@TODO: check if reference to this year is okay or not]
This year marks the 25th anniversary of what is considered to be the first \ac{GPU}, the 
\emph{NVIDIA GeForce 256}. Although graphics chips had been around for a while, especially 
in the professional space of the industry, this card was the first to be marketed as a \ac{GPU}. 

\begin{quote}
    "\emph{What makes the GeForce different than its predecessors is the chip's ability to take over all 
    processing functions for creating three-dimensional graphics. Previously, the computer's main 
    processor would have to share in that responsibility, which could result in slower load times 
    and 'stuttering' on the part of the software.}" \\  
    (CNN Money \cite{CNNMoney1999}, 1999)
\end{quote}

\noindent
Before 1999, the graphics units were specialized chips either used for video encoding and decoding
or expensive hardware targeted for large companies, which arised during the early stages of "3D 
consumer graphics" \cite{Singer2023}. Dedicated and affordable \ac{GPU}s for consumer \ac{PC} had 
their large breakthrough during the 1990s. With the introduction of the \emph{NVIDIA GeForce 256}, 
efficient transformations and lighting computations found their way into private \ac{PC}s 
\cite{Fenno2024}. Back then, this new hardware created a multitude of new possibilities. Some of the 
major innovations featured \emph{cube mapping}, \emph{per-pixel light calculations}, and a 
\emph{standardized vertex buffer} \cite{NVIDIA1999, Battaglia2024}. [@TODO: Re-check vertex buffer]


\noindent
Although it was mostly used for graphics processing - hence the name - nowadays, \ac{GPU}s
are used in a much more general way than in [@TODO: insert year of GPU advent]. Both the 
technical complexity and the fields of application have tremendously increased. Today, a 
\ac{GPU} is often also referred to as a \ac{GPGPU} instead because of its evolution towards 
a multi-purpose tool. \\

\noindent
Nevertheless, \ac{GPU} architecture is still under a strong influence of the entertainment 
industry, first and foremost the games industry. Some of the latest changes to \ac{GPU} 
hardware and \ac{API} design correlates to the ongoing demand for higher output resolutions, 
higher geometric density or additionaly technology for \emph{Deep Learning} algorithms. 
These possibilities led to a lot more applications of \ac{GPU}s in various fields of science, 
like biology, machine learning, video encoding and decoding and more \cite{Battaglia2024}.
To provide more context for how we made use of some of the more modern features of \ac{GPU}s 
we will give a brief overview over the trends in computer graphics over the past decades, and 
especially the last years.


\section{The Rise Of The GPU}

[@TODO: double check games]

\begin{figure}[h]
    \centering
    \includegraphics[width=250px]{images/graphics/bubble-reflection-effects-demo.jpg}
    \caption{The NVIDIA bubble demo showcasing advanced reflections using cube mapping (NVIDIA \cite{NVIDIABubble}, 2000).}
    \label{fig:bubble-reflection-demo}
\end{figure}

\noindent
Before the integration of specialized hardware, games like \emph{Wolfenstein 3D} or the 
original \emph{Doom} made heavy use of \ac{CPU} computations for graphics processing, 
executed sequentially \cite{NVIDIA1999}. A major problem was the demand for increasing 
geometrical detail. Games needed to get bigger, more photo-realistic and provide more dense 
geometry. Increasing the amount of triangles rendered, ultimately led to worse runtime 
performance. The advent of a dedicated, massively parallel chip to perform a large amount of 
operations introduced a solution for this problem. Over the years, more and more stages of the 
rendering pipeline were offloaded to the \ac{GPU}, which in turn resulted in a lot of new effects, 
games with higher triangle density and new lighting technology. As mentioned before, the \emph{NVIDIA 
GeForce 256} introduced cube mapping to the pipeline, featuring real-time reflections, which is 
shown in figure \ref{fig:bubble-reflection-demo} \cite{Battaglia2024}.


\subsection{Rendering Pipeline}

The standardization of the graphics hardware, as pioneered by \emph{NVIDIA}, in concert with capable 
graphics \ac{API}s like \emph{Direct3D} or \emph{OpenGL} kick-started huge graphical improvements in 
games and computer graphics. The heavily parallelized transformations allowed for a lot more triangles 
in games, and the hardware accelerated lighting computations made for even more realistic lighting 
throughout the rendered scenes. One significant advantage: The \ac{GPU} could operate on a large amount 
of data while not stalling the \ac{CPU} \cite{Fenno2024}.\\

\noindent
The new rendering pipeline was to be seen in a lot of games, some of the first being \emph{Epic Games'} 
\emph{Unreal Turnament} (1999) and \emph{id Software's} \emph{Quake III Arena} (1999) \cite{UnrealTurnament, 
Quake3Arena}. \emph{Microsoft's} multimedia \ac{API} \emph{DirectX 7} added support for the new \ac{TL} 
features of \ac{GPU}s. During the following years, more and more features were added, slowly evolving the 
standard towards the "modern" rendering pipeline that is used nowadays. Between the years 1999 and 2009,
20 new minor versions of \emph{DirectX} were released \cite{WikiDirectX}. [@TODO: Check Wikipedia source] 

\begin{figure}[h]
    \centering
    \includegraphics[width=172.5px]{images/graphics/unreal-turnament.jpg}
    \includegraphics[width=230px]{images/graphics/quake-iii-arena.jpg}
    \caption{A screenshot from the game \emph{Unreal Turnament} (1999) (left) and \emph{Quake III Arena} (1999) 
    (right) \cite{GamespotUnrealTurnament, GameWatcher2006}.}
    \label{fig:unreal-turnament-quake-arena}
\end{figure}


\subsection{Deferred Rendering}

With a lot of innovative hard and software, the games got complexer and real-time computer graphics got even more 
photo-realistic. The demand for more light sources increased in an attempt to make game worlds more realistic. 
However, more light sources posed some problems for the developers. When using the traditional rendering pipeline,
lighting is calculated on a per-vertex basis. Interpolation can be used to generate lighting results for every sample 
in between the vertices. Still, every vertex must be considered in combination with all light sources. This creates a 
dependancy between the amount of vertices and the amount of lightsources. Basically, adding a few more light sources 
will drastically increase computatoin times of the final image. Another method can be adopted: \emph{Deferred Shading}.
It was first adopted for the use on \ac{GPU}s by Dean Calver \cite{Calver2004} in 2004. This technique makes use of 
multiple render targets, and writes the data necessary for the lighting calculations into various buffers. This way, 
the surface normals, the specular intensity, the albedo (texture color), the depth and more values can all be stored 
individually, as per-pixel data. This includes the relevant data for the lighting calculations. When all buffers are 
drawn, the final result is easily found by looking up all the relevant data from each buffer at one pixel coordinate 
and combining them to a final result. Figure \ref{fig:deferred-shading-buffers} shows all the buffers used for the 
creation of one frame in \emph{Guerilla Games'} \emph{Killzone 2} (2007) \cite{KillzoneFandom}. 

\begin{figure}[h]
    \centering
    \includegraphics[width=175px]{images/graphics/killzone-2-buffer-depth.png}
    \includegraphics[width=175px]{images/graphics/killzone-2-buffer-vsn.png}
    \includegraphics[width=175px]{images/graphics/killzone-2-buffer-specular.png}
    \includegraphics[width=175px]{images/graphics/killzone-2-buffer-specular-rough.png}
    \includegraphics[width=175px]{images/graphics/killzone-2-buffer-ss-motion.png}
    \includegraphics[width=175px]{images/graphics/killzone-2-buffer-albedo.png}
    \includegraphics[width=175px]{images/graphics/killzone-2-buffer-composed-result.png}
    \includegraphics[width=175px]{images/graphics/killzone-2-buffer-post.png}
    \caption{Different buffers for \emph{Deferred Shading} in \emph{Guerilla Games'} \emph{Killzone 2} (2009) 
    \cite{Valient2007}.}
    \label{fig:deferred-shading-buffers}
\end{figure}

\noindent
As Akenine-Möller et al. point out, "the contributionof the models’ geometry has been fully decoupled from lighting 
computations" \cite{AkenineMoeller2018}. \emph{Deferred Shading} encompasses a lot more optimizations, like compression, 
light clustering, deferred texturing and more \cite{AkenineMoeller2018}. All these techniques combined can ultimately 
result in a more efficient shading model for specific use cases, for instance, when using a lot of lights in combination 
with high density geometry.\\

\noindent
The basic idea of \emph{Deferred Shading} influenced the next couple of years of real-time graphics technology and is 
still used today for state-of-the-art game development. Dispatching work for the \ac{GPU} to operate on and optimizing 
memory and data layouts for most efficient computation has also resulted in one of the latest trends in real-time 
rendering.

\subsection{GPU Driven Rendering}

Over a decade after the first \ac{GPU} was introduced, hardware acceleration was the de facto standard in 
computer graphics and the rendering pipeline got more and more loaded with features, innovations and data. 
This led to a new bottleneck: The memory bandwith between the \ac{CPU} and \ac{GPU}. Soon, a new trend was 
developed: . The goal of  is, to minimize bandwidth between \ac{CPU} and \ac{GPU}. 


\section{Hier kommt meine Motivation und Überleitung zur Forschungsfrage} \label{sec-todo}

The latest innovations in computer graphics have been developing in different directions,
one of them being \ac{GPU} Driven Rendering. Minimizing dependencies between \ac{CPU} and \ac{GPU} 
has been an ongoing effort for the last decade. Many of the novel approaches which have been developed 
over the years are visible in the latest technology. But there are still plenty of use cases 
which are not yet living up to their full potential, considering modern hard- and software solutions.\\

\noindent
One of those use cases is Voxel Rendering. Although there are ongoing efforts to improve voxel 
rendering performance, we believe that new innovations can be applied to this field of computer 
graphics. To highlight the possible improvements, one can consider different applications of 
voxel rendering. 






\section{The Idea}

[@TODO: Mention initial idea]

\noindent
In the next chapter we will focus on the first use case, the Rasterized Voxel Rendering. We belive that this 
technique can benefit the most from advances in rendering technology.
With most of the latest improvements of graphics hard- and software, the capabilities to process mesh data 
and compute a rasterized image got better and better. Even the graphics hardware vendor's efforts to enable 
real-time ray traced lighting, shadows, reflections and ambient occlusion is applied within the rasterized 
pipeline and therefore relies on improvements to said pipeline over the last years. \\
We will first lay out a technical foundation in chapter \ref{cpt-technical-background} before discussing 
related work on which our approach relies on in chapter \ref{cpt-related-work}.



- Spectrum between CPU computation and GPU computation. 
CPU <-----------> GPU

- Mesh-Shading pipeline as new \st{standard} pipeline

- Poses question, can mesh shading further optimize voxel rendering?

- Voxel Rendering

- Better geometry creation (in contrast to Geometry Shader)
- Optimized occlusion culling using per-meshlet OC
