\chapter{Conclusions} \label{cpt-conclusion}

This work has shown that the proposed implementation, a customization of the 
\ac{TPOC} algorithm, enables \ac{GPU}-driven occlusion culling for dense, volumetric 
voxel scenes. The measurments have shown the capability of the modern Mesh Shading 
rendering pipeline to increase culling efficiency by culling individual meshlets, 
which in turn leads to a lower computational load for the rasterizer and pixel shader 
stages. The \ac{GPU} experiment has provided a precise measurement that indicates a 
performance increase of about $29 \%$ when using the per-meshlet culling configuration 
of the algorithm as compared to the more general per-octree node occlusion culling 
configuration. \\

\noindent
The measurments have provided an overview of culling efficiency depending on several aspects 
of the algorithm's configuration. The model's characteristics have a significantly 
effect on the culling efficiency, which increases for models with high volumes and flat 
surfaces, and decreases for small, thin, and detailed geometry with holes or empty spaces 
within the model. \\

\noindent
Although several million voxels were computed by the pipeline each frame, the fast and 
\ac{GPU}-driven algorithm scales with the amount of visible voxels and partially scales 
with screen resolution. A considerable part of the overhead is static in its computation 
and can be easily compared to the performance gain. Consequently, the approach can be 
implemented for suitable models and is expected to cull a significant number of voxels, 
reducing the pipeline's computation times on the \ac{GPU} side and decrease \ac{CPU} load 
simultaneously. \\

\noindent
The per-meshlet occlusion culling has been proven to be suitable for the culling algorithm.
It provides good results and is compatable with the \ac{GPU}-driven pipeline that enables 
on-chip creation of millions of meshlets within very low execution times. \\

\noindent
Ultimately, this work has shown that voxel rendering can benefit from the latest advances 
in real-time rendering and that the use of the proposed occlusion culling algorithm can 
optimize the runtime performance as compared to a similar per-octree culling approach, or 
compared to not using any occlusion culling at all. [@TODO: Only say that if tests were done]


