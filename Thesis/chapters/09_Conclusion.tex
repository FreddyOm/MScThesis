\chapter{Conclusions} \label{cpt-conclusion}

This work has presented a implementation for a \ac{GPU}-driven implementation of 
\ac{HZB} for voxel-based volumetric rendering. A spatial data structure is 
used to approximate the scenes inner volume and replace vast amount of voxels 
with a small amount of cubical geometry. This way, a set of best occluders can 
be used for computing a depth prepass, which in turn is used for efficient 
occlusion culling on the \ac{GPU}. An examplary voxel scene with over 3 million 
voxels has been approximated by only 8256 larger voxels.  \\

\noindent
The measurments have shown the capability of the modern mesh shading rendering 
pipeline to increase culling efficiency by culling individual meshlets, which in 
turn leads to a lower computational load for the rasterizer and pixel shader 
stages. The \ac{GPU} experiment has provided a precise measurement that indicates a 
performance increase of about $29 \%$ when using the \ac{PMOC} configuration 
of the algorithm as compared to the more general \ac{PONOC} configuration. 
Compared to the base pipeline without occlusion culling apllied to the scene, the 
performance increase ranged from $61.3 \%$ for the highly complex and irregular 
\emph{Hairball} scene up to $90.7 \%$ for the \emph{Bunny} scene. \\

\noindent
The measurments have provided an overview of culling efficiency depending on several aspects 
of the algorithm's configuration. The model's characteristics have a significantly 
effect on the culling efficiency, which increases for models with high volumes and flat 
surfaces, and decreases for small, thin, and detailed geometry with holes or empty spaces 
within the model. The approach is therefore suggested for scenes that are dense and tight, 
and can be expected to provide at least a reasonable amount of full octree nodes. Conversely, 
low voxel resolutions even tend to result in worse performance for this algorithm because of 
the lower probability of featuring full octree nodes. \\

\noindent
Although several million voxels were computed by the pipeline each frame, the fast and 
\ac{GPU}-driven algorithm scales with the amount of visible voxels and partially scales 
with screen resolution. A considerable part of the overhead is static in its computation 
and can be easily compared to the performance gain. Consequently, the approach can be 
implemented for suitable models and is expected to cull a significant number of voxels, 
reducing the pipeline's computation times on the \ac{GPU} side and decreasing \ac{CPU} 
load simultaneously. \\

\noindent
Furthermore, the pixel shader overdraw was shown to be considerably reduced by using 
the \ac{PMOC} as compared to no occlusion culling at all or even the \ac{PONOC} 
configuration. Oversraw is mostly present on the surface of the scene, while being 
greatly reduced for otherwise ineffecient camera angles. \\

\noindent
The measurments for querying the best occluders have also shown that the approach can 
generally be used to update the voxel data dynamically in real time. This is a key 
requirement for the use of the algorithm in real-time voxel-rendered games since 
manipulating the voxel data arbitrarily is a common feature in such games. The 
measurements have shown that the \ac{CPU} computations required for querying the 
best occluders is within reasonable time budgets. The computations are expected to 
be further optimized by chunking voxel data and using multithreaded computations for 
the update and query. Nevertheless, it is up to future work to do an in-depth evaluation.

\noindent
Ultimately, this work has shown that voxel rendering can benefit from the latest advances 
in real-time rendering and that the use of the proposed occlusion culling algorithm 
using modern mesh shading can optimize the runtime performance as compared to a similar 
per-octree culling approach, or compared to not using any occlusion culling at all. \\

\noindent
The approach could be used in large voxel scenes, which are meant to feature both a 
high-resolution voxel scene, and a dynamic environment capable of complex player or 
physical interaction. It is also possible to use it to animate high-resolution voxel 
models or even to compute virtualized geometry. The latter case could feature a 
continously adapting geometrical resolution depending on the camera distance and 
could provide similar results to Unreal Engine's \emph{Nanite virtualized geometry} 
system \cite{Karis2021}. In their presentation, Karis et al. mention that they 
considered to approach virtualized geometry by using voxels. 