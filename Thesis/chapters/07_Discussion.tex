\chapter{Discussion} \label{cpt-discussion}

Finally, possible advantages and disadvantages of the proposed approach are discussed based on the data
presented in chapter \ref{cpt-experiment}. Other occlusion culling algorithms can be considered for 
qualitative comparison to this approach, although a final comparison would require quantitave data to 
support any assumptions.\\






%
%- Pro: 
%    - No meshlet precomputation, since meshlets are created on the GPU
%    - Meshlet voxels are inherently AABBs which is quite helpful for intense intersection computations % @TODO: Check if there are even any intersection comps to be done
%    - Cores of most models can be "removed"
%    - New possibility of using occluders for occlusoin of the same "mesh"
%- Con:
%    - Depending on the voxel size, there might be a need for chunking the data effectively % @TODO: Check if this is still necessary with Meshlet Occlusion Culling enabled
%    - Voxels as meshlets -> non-similar normals can be inefficient.. 
%    - Note that the amount of threads used per group is equal to the amount of voxels within the repsective octree 
%    node, leaving a significant amount of threads unused for scarcely populated octree nodes.
%    - Memory bandwidth between CPU and GPU is not really changed. Only the scheduling of the draw calls is optimized.
%    Works fine in practice but is very use case sensitive.
%
%Discussion points:
%
%- When already drawing aggregated bounding boxes, isn't this just HiZ Buffering? And is it then better to draw everything
% to test for visibility instead of only best occluders? 